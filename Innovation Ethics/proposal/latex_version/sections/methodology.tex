\section{Methodology}

This research employs a mixed-methods approach \parencite{creswell2017} to investigate how platforms implement and characterize optimization systems. The methodology specifically addresses the misalignment between innovation rhetoric and optimization reality through three complementary analytical streams: (1) discourse analysis of platform communications and implementation documents, (2) comparative analysis of system boundaries pre- and post-implementation, and (3) stakeholder experience assessment.

The core dataset comprises four longitudinal case studies of major platform organizations. Each case study begins with systematic documentation of how the initiative is framed and communicated, analyzing platform narratives, marketing materials, and internal documents for innovation-related claims. These claims are then evaluated against \textcite{march1991}'s theoretical distinction between optimization and innovation through structured assessment of:
\begin{enumerate}[label=(\alph*)]
    \item whether the initiative operates within or creates new system boundaries,
    \item whether it maximizes existing metrics or establishes new value definitions, and
    \item whether it constrains or expands stakeholder possibility spaces.
\end{enumerate}

Stakeholder impacts are measured through multiple instruments. Semi-structured interviews ($n=120$) employ protocol analysis to identify gaps between innovation expectations and optimization experiences. Social network analysis maps structural changes in platform relationships, with particular attention to whether changes represent optimization of existing networks or innovative reconfigurations. A novel ``Innovation-Optimization Alignment Index'' (IOAI) synthesizes these measures, providing quantitative assessment of the degree and impact of misalignment in each case. 